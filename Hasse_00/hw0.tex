\documentclass[a4paper]{article}
\usepackage[letterpaper, margin=1in]{geometry} % page format
\usepackage{listings} % this package is for including code
\usepackage{graphicx} % this package is for including figures
\usepackage{amsmath}  % this package is for math and matrices
\usepackage{amsfonts} % this package is for math fonts
\usepackage{tikz} % for drawings
\usepackage{hyperref} % for urls
\usepackage{ifthen}
 \usepackage{setspace} 

\newcommand\myqed{}                 % creates command for tombstone at end of proof
\newcommand{\printmyqed}[1][]       % decides whether to print tombstone or not
   {%
   \ifthenelse{\equal{#1}{Proof}}
   {\renewcommand{\myqed}{\qed}}
   {\renewcommand{\myqed}{}}
   }

\newenvironment{exercise}[1][]{%
  \bigskip                          % Space before problem statement
  \noindent \textsf{Exercise #1.\quad}\slshape }{}
     
\newenvironment{response}[1][\textit{Solution}]{%
  \printmyqed[#1]
  \begin{spacing}{\spacingfactor}
  \medskip                          % Space before solution 
  \noindent \textit{#1.\quad}}{\myqed\end{spacing}\medskip\hrule}


\title{Homework \LaTeX Template}
\author{Dr. Pablo Rivas}
\date{5/8/16}

\begin{document}
\lstset{language=Python}

\maketitle

\section{Solution to Problem 1}
The derivative of $g(x) = -3x^2 + 24x + - 30$ is equal to $-6x +24$.
To find the value that maximizes this function we set $-6x + 24 = 0$ and solve for $x$. After solving for $x$ we get $x=4$, which is the value for x that maximizes the function $g(x) = -3x^2 + 24x + - 30$.
\section{Solution to Problem 2}
The derivative with respect to $x_0 = 9x^2_0 - 2x^2_1$. \\
The derivative with respect to $x_1 = -4x_0x_1 + 4$.

\section{Problem 3}
\begin{exercise}[3a]
It is not possible to multiply these two matrices.
\end{exercise}

\begin{exercise}[3b]
\[ A^T=
			\begin{bmatrix}
			1 & 2 \\
			4 & -1 \\
			-3 & 3
			\end{bmatrix}
\]
\[A^TB=
			\begin{bmatrix}
			-2 & -2 & 13 \\
			-8 & 1 & 16 \\
			6 & -3 & -3
			\end{bmatrix}
\]
$A^TB$ has a rank of 2.
\end{exercise}

\section{Solution to Problem 4}
\textbf{Simple Gaussian}: A function of the form: $f(x) = ae^{-\frac{(x-b)^2}{2c^2}}$ \\
\textbf{Multivariate Gaussian}:  A random vector is said to be k-variate normally distributed if every linear combination of its k components has a univariate normal distribution. \\
\textbf{Bernoulli Distribution}: A probablity distribution of a random variable which takes the value 1 with success probability of $p$ and the value $0$ with failure probability of $q=1-p$. \\
\textbf{Binomial Distribution}: A distribution giving the probability of obtaining a specified number of successes in a finite set of independent trials in which the probability of a success remains the same from trial to trial. \\
\textbf{Exponential Distribution}: A distribution that is used to model time between the occurrence of events in an interval of time, or the distance between events in space.


\section{Solution to Problem 6}
The expected value of for this variable is $2.5$.

\section{Solution to Problem 7}

\begin{exercise}[7a]
The set $z$ is in $1$ dimention, therefore we get $d = x - 1.1$. the smallest number for $x^*$ we can get while $x \in N$ is $1$. Therefore $x^* = 1$.
\end{exercise}

\begin{exercise}[7b]
$x^*$ would be located on the closet edge (the shortest linear distance) of $Z$ to $y$.
\end{exercise}

\section{Solution to Problem 8}
\begin{exercise}[8a]
Because the random variable distribution is $e^-y$ for all positive values, it proves that the value will be $1$.
\end{exercise}

\footnotesize
\begin{thebibliography}{99}

  \bibitem{HB98} Huynen, M.~A. and Bork, P. 1998. Measuring genome evolution.
    \emph{Proceedings of the National Academy of Sciences USA} 95:5849--5856.

\end{thebibliography}

\end{document}
