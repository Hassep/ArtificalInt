\documentclass[a4paper]{article}
\usepackage[letterpaper, margin=1in]{geometry} % page format
\usepackage{listings} % this package is for including code
\usepackage{graphicx} % this package is for including figures
\usepackage{amsmath}  % this package is for math and matrices
\usepackage{amsfonts} % this package is for math fonts
\usepackage{tikz} % for drawings
\usepackage{hyperref} % for urls
\usepackage{ifthen}
 \usepackage{setspace} 

\newcommand\myqed{}                 % creates command for tombstone at end of proof
\newcommand{\printmyqed}[1][]       % decides whether to print tombstone or not
   {%
   \ifthenelse{\equal{#1}{Proof}}
   {\renewcommand{\myqed}{\qed}}
   {\renewcommand{\myqed}{}}
   }

\newenvironment{exercise}[1][]{%
  \bigskip                          % Space before problem statement
  \noindent \textsf{Exercise #1.\quad}\slshape }{}
     
\newenvironment{response}[1][\textit{Solution}]{%
  \printmyqed[#1]
  \begin{spacing}{\spacingfactor}
  \medskip                          % Space before solution 
  \noindent \textit{#1.\quad}}{\myqed\end{spacing}\medskip\hrule}


\title{Homework 02}
\author{Philip Hasse}
\date{9/29/16}

\begin{document}
\lstset{language=Python}

\maketitle

\section{Solution to Problem 2.1(a)}
The equation to find the sample size, $N$, with a complexity of $M=1$ is $\epsilon(M,N,\delta) = \sqrt{\frac{1}{2N}\ln(\frac{2(1)}{.03})} \le .05$.
\begin{align}
&= \frac{1}{2N}\ln(\frac{2(1)}{.03}) \le .0025 \\
&= \ln(\frac{2}{.03}) \le .005N \\
&= \frac{\ln(\frac{2}{.03})}{.005} \le N \\
&= N \ge 840
\end{align}
\section{Solution to Problem 2.1(b)}
Simlar to the previous question we use the equation to find the sample size, $N$, with a complexity of $M=100$ this time.
\begin{align}
 \epsilon(M,N,\delta) &= \sqrt{\frac{1}{2N}\ln(\frac{2(100)}{.03})} \le .05 \\
&= \frac{1}{2N}\ln(\frac{2(100)}{.03}) \le .0025 \\
&= \ln(\frac{200}{.03}) \le .005N \\
&= \frac{\ln(\frac{200}{.03})}{.005} \le N \\
&= N \ge 1761
\end{align}

\section{Solution to Problem 2.1(c)}
Simlar to the previous question we use the equation to find the sample size, $N$, with a complexity of $M=10000$ this time.
\begin{align}
 \epsilon(M,N,\delta) &= \sqrt{\frac{1}{2N}\ln(\frac{2(10000)}{.03})} \le .05 \\
&= \frac{1}{2N}\ln(\frac{20000}{.03}) \le .0025 \\
&= \ln(\frac{20000}{.03}) \le .005N \\
&= \frac{\ln(\frac{20000}{.03})}{.005} \le N \\
&= N \ge 2683
\end{align}

\section{Solution to Problem 2.11(N=100)}
To find the the outside error we use the equation $E_{out}(g) \le E_{in}(g) + \frac{8}{N}\ln(\frac{m_H(2N)}{\delta})$.
Plugging our given values we get $E_{out}(g) \le E_{in}(g) + \frac{8}{100}\ln(\frac{4(200+1)}{.1}) = E_{in}(g) + 0.719$


\section{Solution to Problem 2.11(N=10000)}
Similar to the last problem, to find the the outside error we use the equation $E_{out}(g) \le E_{in}(g) + \frac{8}{N}\ln(\frac{m_H(2N)}{\delta})$.
Plugging our given values we get $E_{out}(g) \le E_{in}(g) + \frac{8}{10000}\ln(\frac{4(20000+1)}{.1}) = E_{in}(g) + 0.011$

\section{Solution to Problem 2.12}
To find the sample size, $N$, with the given information we use the equation $N \ge \frac{8}{\epsilon^2}\ln(\frac{4((2N)^{d_{vc}}+1)}{\delta})$.
Plugging in our given values we get:
\begin{align}
N \ge \frac{8}{.05^2}\ln(\frac{4((2N)^10+1)}{.05}) &= 3200N \ge \ln(\frac{4((2N)^10+1)}{.05}) \\
&= N \ge 452957
\end{align}

\section{Solution to Problem 3.1}
\begin{figure}
  \includegraphics[width=\linewidth]{31(a).jpg}
  \caption{3.1}
  \label{fig:3.1(a)}
\end{figure}

\begin{figure}
  \includegraphics[width=\linewidth]{31(b).png}
  \caption{3.1}
  \label{fig:3.1(b)}
\end{figure}



\end{document}
